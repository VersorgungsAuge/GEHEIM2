\section{Durchführung}
\label{sec:Durchführung}
\begin{figure}[h]
	\centering
	\includegraphics[width=14cm,height=11cm]{fotos/Bilddurchfuhrung1.pdf}
	\caption{Schematischer Aufbau der Messapparatur}
	\label{durch:1}
\end{figure}
Zuerst soll eine Vorbereitungsaufgabe gelöst werden, um mit der Durchführung beginnen zu können.
Als erstes wird die Cd-Lampe mit einem Objektiv und einer Linse $L1$ so eingestellt, dass sie scharf auf den Spalt $S1$ abbildet. Im nächsten Schritt wird die Linse $L2$ justiert, sodass ein paralleles Lichtbündel auf das Glasprisma fällt. Um Strahlungsverluste zu vermeiden, wird der Durchmesser des Lichtbündels möglichst klein gehalten. Desweiteren soll der erste Schritt aus der Durchführung für die Linse $L3$ und den Spalt $S2$ wiederholt werden. Es soll nun möglich sein mit diesem Bild eine Wellenlänge auszuwählen. Nun soll mittels der Linse $L4$ ein scharfes Bild auf die Lummer-Gehrcke-Platte abgebildet werden. Beim nächsten Schritt wird die Polarisation je nach Versuchsteil auf $0^{\circ}$ bzw. $90^{\circ}$ eingestellt. Zuletzt wird das Magnet eingeschaltet um die Zeeman-Linien zu beobachten. Als letztes wird das Bild mit einer Digitalkamera aufgenommen und gespeichert.


